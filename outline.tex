\documentclass{article}

\usepackage[english]{babel}
\usepackage[utf8]{inputenc}
\usepackage{fancyhdr}
\usepackage{xcolor}
\usepackage{helvet}
\usepackage{setspace}
\usepackage{graphicx}
\usepackage[hyphens]{url}
\usepackage[hidelinks]{hyperref}
\usepackage[left=18mm,right=18mm,top=20mm,bottom=40mm]{geometry}
\usepackage{titling}
%\setlength{\droptitle}{-25mm}

\setlength{\headheight}{25mm}
%\setlength{\footskip}{5mm}

\pagestyle{fancy}
\fancyhf{}
\lfoot{\textcolor{gray}{Course Outline, Technology \& Innovation}}
\lhead{{\includegraphics[width=120mm]{scs.png}}}
\rfoot{\textcolor{gray}{!!CODE!! (Section !!SECTION!!),  Page {\thepage} / \pageref{LastPage}}}

\setstretch{1.3}
\setlength{\parskip}{6pt}
\setlength{\parindent}{0pt}

\title{Course outline for !!CODE!! \\ {\sc !!NAME!!}}
\date{Section !!SECTION!! for !!TERM!! \\ !!CREDITS!!}
\author{{\bf McGill University School of Continuing Studies} \\
  {\em Technology \& Innovation} \\
  Taught by !!INSTRUCTOR!! }
\renewcommand{\familydefault}{\sfdefault} 

\begin{document}


\maketitle

\thispagestyle{fancy}

!!ERROR!!

\newpage

\setstretch{0.9}
\tableofcontents
\setstretch{1.2}

\newpage

\section{Course Information}

\begin{description}
\item[Office hours]{ !!HOURS!!}
\item[Assistants]{ !!ASSISTANT!! }
\item[Description]{ !!DESCRIPTION!! }
\item[Pre-requisites]{!!PREREQ!!}
\item[Co-requisites]{!!COREQ!!}
\item[Contact hours]{!!CONTACT!!}
!!ASSIGNMENT!!
\item[Learning outcomes]{ !!OUTCOMES!! }
\item[Instructional methods]{ !!METHODS!! }

\end{description}

Note that, in accordance with McGill University’s official email
policy, the instructor {\em will not respond} to email messages sent
from {\bf outside} the McGill network (e.g., gmail or hotmail). As
well, while instructors appreciate that email correspondence is an
efficient method of communication, students should keep in mind that
instructors arenot on call. Students should therefore not expect a
response to emails during the weekend or after 5:00 p.m., Mondays
through Fridays.

Prior to contacting instructors via email, students seeking
course-related information, including information about assignments,
should first attempt to locate that information by (a) consulting the
course syllabus or textbook, (b) consulting materialsuploaded to the
course website (myCourses), or (c) consulting with other students in
the class. Generally speaking, during the week, a valid request
submitted via e-mail will receive a response within 36 hours,
exclusive of weekends and statutory holidays.

\vfill

\hrule
 
{\em The official version of this course outline is the version posted
  on myCourses or, if any, the printed version distributed by the
  instructor at the beginning of the course.}

{\em In the event of extraordinary circumstances beyond the
  University's control, the content and/or evaluation scheme in this
  course is subject to change.}


\newpage

\subsection{Materials}

\subsubsection{Required Materials}

!!REQUIRED!!

!!OPTIONAL!!

\newpage

\subsection{Evaluation}
\label{eval}

\begin{center}
\scalebox{0.8}{
  \begin{tabular}{p{15mm}|p{30mm}|p{30mm}|p{100mm}}
    \% & Item & Deadline & Description \\
    \hline
  !!ITEMS!!
\end{tabular}}
\end{center}

\newpage

\subsection{Course Content}

\begin{enumerate}
\item{!!CONTENT!!}
\end{enumerate}

\newpage

\section{Important Information}

\subsection{Land Acknowledgment}

McGill University is located on land, which has long served as a site
of meeting and exchange amongst Indigenous peoples, including the
Haudenosaunee and Anishinabeg nations. McGill honours, recognizes and
respects these nations as the traditional stewards of the lands and
waters on which we meet today.

\subsection{McGill Resources for Academic Success}

\begin{description}
\item[Inclusive Learning Environment]{McGill is committed to providing
  an inclusive and supportive learning environment. If you experience
  barriers to learning in this course, do not hesitate to discuss them
  with your instructor. If you have a special learning need or
  disability, you are encouraged to contact the Office for Students
  with Disabilities. For more information, please visit
  \url{https://mcgill.ca/osd/}.}
\item[Health and Wellness]{Student well-being is a priority for the
  University, the School (SCS), and the McGill Association of
  Continuing Education Students (MACES). Should you find yourself in
  need of support, please keep in mind that there are a number of
  resources available to help you. Many SCS students are automatically
  covered by the MACES Health and Dental Plan. For further details,
  please visit \url{https://maces.ca/}. In addition, eligible students
  will also be covered by a virtual healthcare service provided by
  {\em Dialogue} through MACES. The Dialogue service allows students
  to connect virtually with nurses and physicians in Canada via a
  mobile or web app. Both the MACES Health and Dental Plan and the
  Dialogue app include access to professional psychologists. MACES
  students also currently have free access to \texttt{Keep.meSAFE}, a
  psychological counselling service where students can speak to a
  counsellor in one of six languages. For more information, please
  visit
  \url{https://www.mcgill.ca/continuingstudies/student-services}.}
\item[McGill Writing Centre]{Writing well is key to both academic and
  professional success. The {\em McGill Writing Centre} (MWC) offers
  credit courses in academic and professional writing, and a tutorial
  service open to all McGill students. The tutorial service offers
  one-to-one sessions with seasoned instructors and experienced tutors
  who will work with you at any stage of the writing process.  For
  information about the availability of in-person and online
  appointments, please visit
  \url{https://www.mcgill.ca/mwc/tutorial-service}.}
\item[McGill Library]{Find a workshop, learn about library services,
  and reach out to your liaison librarian for research help at
  \url{https://www.mcgill.ca/library/orientation}.}
\end{description}

\newpage

\subsection{Academic Conduct}

\begin{description} 
\item[Academic Integrity]{Students are responsible for knowing
  McGill's rules and regulations concerning academic honesty, which
  can be found on the Student Rights and Responsibilities
  website. Violations of academic integrity undermine not only the
  value of honest students' work, but also the academic integrity of
  the University and the value of a McGill credential. The Student
  Rights and Responsibilities website provides resources that can help
  students avoid dishonest work, and an explanation of the
  disciplinary measures that go with it. To learn more about academic
  integrity, visit \url{https://www.mcgill.ca/students/srr/honest}.
  {\bf All newly admitted students must complete the Academic
    Integrity Tutorial (AIT) in Minerva during their first semester at
    McGill.} Failure to complete the tutorial will restrict the
  student from registering for courses in the following semester. The
  Tutorial can be accessed as follows: \texttt{Minerva / Student Menu
    / Academic Integrity Tutorial}.}
\item[Respectful and Professional Communication]{This course is
  designed to help you learn to communicate professionally both during
  your time at McGill and in your future workplaces. In keeping with
  McGill's policies on student rights and responsibilities, it is
  expected that during class discussions and small group interactions
  you will communicate constructively and respectfully. Sexist,
  racist, homophobic, ageist, and ableist expressions will not be
  tolerated in the classroom or during group meetings held outside of
  class.}
 \end{description}

To learn more about these policies, please consult 
\url{https://mcgill.ca/students/srr/policies-student-rights-and-responsibilities}.

Students {\bf may not record any class proceedings} or collect any
electronic data (including photos and videos) from class activities
without the express consent of the instructor. Instructor generated
course materials (e.g., handouts, notes, summaries, test questions,
etc.) are protected by law and may not be copied or distributed in any
form or in any medium without the explicit permission of the
instructor. Note that infringements of copyright can be subject to
followup by the University under the {\em Code of Student Conduct and
Disciplinary Procedures}.

\subsection{Policy Against Sexual Harassment and Violence}

McGill University is committed to creating and sustaining a safe
environment through proactive, visible, accessible, and effective
approaches that seek to prevent and respond to sexual harassment and
sexual violence. McGill's Policy against Sexual Violence underlines
this commitment and ensures that procedures are in place to address
complaints. To learn more, visit \url{https://www.mcgill.ca/osvrse/}.

\paragraph{Sexual Violence Training {\em It Takes All of Us}}

In accordance with Québec law (Bill 151), all newly admitted students
must complete the sexual violence training course, called ``It Takes
All of Us,'' in myCourses during their first semester at
McGill. Students will receive an automated email letting them know
when they have been enrolled in the training course. Failure to
complete the training will restrict the student from registering for
courses in the following semester. The training can be accessed
through myCourses and will appear as a separate course in your
profile. For more information, please visit \url{https://www.mcgill.ca/osvrse/}.

\subsection{Policy on Harassment and Discrimination}

McGill University is committed to promoting an equitable environment
where the fundamental dignity of all of its members is respected. The
objectives of {\em McGill's Policy on Harassment and Discrimination
Prohibited by Law} are to promote education and awareness about equity
issues and to ensure that procedures are in place to address
complaints. To learn more about McGill's policy, including how to
report a complaint, please visit
\url{https://www.mcgill.ca/how-to-report/}.

\subsection{Right to Submit in English or French}

In accord with McGill University's {\em Charter of Students' Rights},
students in this course have the right to submit in English or in
French any written work that is to be graded. This does not apply to
courses in which acquiring proficiency in a language is one of the
objectives.

Please reach out to the instructors and the assistants to inquire if
they are able to additionally advise you in other languages,
remembering that they are not required to do so. If you do share a
language, there is no impediment to using that in informal
discussions, as long as submitted work and any written agreements and
complaints are either in English or in French.

\subsection{Grades}

\subsubsection{Final Grades}

The official final course grade is the one that appears in the {\em
  Athena Student Portal} on your Record of Study. A final grade
appearing in other locations (for example, myCourses) may be subject
to change.  The School of Continuing Studies reserves the right to
correct mistakes.

\subsubsection{Undergraduate Credit Course Grading System}

\begin{center}
  \begin{tabular}{ccl}
     Result & Numerical Scale (\%) & Letter Grade \\
   \hline
Pass & 85--100 & A  \\
& 80--84 & A- \\
& 75--79 & B+ \\
& 70--74 & B \\
& 65--69 & B- \\
& 60--64& C+ \\
& 55--59 & C \\
& 50--54 & D \\
\hline
Failure & 0--64 & F 
  \end{tabular}
  \end{center}

Although a D is a passing grade, it {\em will not permit entry} into a
subsequent course for which it is aprerequisite, and nor will it be
recognized if the course is a required course in your program. You
{\bf must} obtain a grade of C or better in courses that you take to
fulfil program requirements. Youmay not register in a course unless
you have passed all the prerequisite courses with a grade of C or
better, except by written permission of the appropriate department
chair.

\subsubsection{Graduate Credit Course Grading System}


\begin{center}
  \begin{tabular}{ccl}
     Result & Numerical Scale (\%) & Letter Grade \\
   \hline
Pass & 85--100 & A  \\
& 80--84 & A- \\
& 75--79 & B+ \\
& 70--74 & B \\
& 65--69 & B- \\
\hline
Failure & 0--64 & F
  \end{tabular}
  \end{center}

You must obtain a grade of B- or better in courses that you take to
fulfil program requirements. Youmay not register in a course unless you
have passed all the prerequisite courses with a grade of B- or better,
except by written permission of the appropriate department chair.

\subsubsection{Non-Credit Course Grading System}

\begin{center}
  \begin{tabular}{ccl}
     Result & Numerical Scale (\%) & Letter Grade \\
   \hline
Pass & 85--100 & A  \\
& 80--84 & A- \\
& 75--79 & B+ \\
& 70--74 & B \\
& 65--69 & B- \\
\hline
Failure & 0--64 & F
  \end{tabular}
  \end{center}

You must obtain a grade of B- or better to fulfil program
requirements. You may not register in a course unless you have passed
all the prerequisite courses with a grade of B- or better.

\paragraph{What does a final course grade of $J$ mean?}

A $J$ grade is a failing grade due to either (i) an unexcused absence
for an official final exam, or (ii) failure to submit required work
worth more than 20\% of the final grade for the course as a whole. A
$J$ grade is calculated as a failure.

To learn more about University letter grades, visit
\url{https://www.mcgill.ca/study/20182019/university_regulations_and_resources/continuing/scs_gi_student_records_grading_grade_point_average}.

\newpage

\subsection{Assessments}

Work submitted for evaluation as part of this course {\bf may be
  checked with text-matching software} within myCourses.

To learn more about assessment of student work, consult McGill's
Student Assessment Policy at
\url{https://mcgill.ca/secretariat/files/secretariat/2016-04_student_assessment_policy.pdf}.


\subsubsection{Assignments}

Please refer to Section \ref{eval} concerning assignment submission
for this course. Unless this outline or another written document from
your instructor authorizes you to {\bf share} the assignment results
online or directly, please keep the questions and the responses {\em
  confidential}.

Legitimate exceptions for late submission of assignments include
documented medical, family, and personal emergencies, and observance
of holy days. Students planning to observe holy days (see
\url{https://www.mcgill.ca/importantdates/holy-days-0/policy-holy-days})
listed in the McGill calendar should notify the instructor by email no
later than two (2) weeks prior, and preferably at the beginning of the
course.

\subsubsection{Examinations}

If the course includes exams, students should not make other
commitments during a scheduled exam, which is indicated on your course
outline. Vacation plans do not constitute valid grounds for the
deferral or the rescheduling of examinations, tests or
assignments. See the School of Continuing Studies Calendar for the
regulations governing examinations, or go to
\url{https://www.mcgill.ca/continuingstudies/current-students/exams}.

Students who have a documented disability and require academic
accommodation must contact McGill's Office for Students with
Disabilities (OSD). For information on Exam Accommodation, visit
\url{https://www.mcgill.ca/osd/exams} and
\url{https://mcgill.ca/osd/student-resources/forms/scs-accommodation-request}.


Students may {\em request a deferral} of final examinations or timed
tests and assignments for medical reasons or out-of-town business
commitments, which must be justified in writing with a medical
certificate or company letter. To learn more, visit
\url{https://www.mcgill.ca/continuingstudies/exams-conflicts-deferrals-and-rereads}.


Students requesting the {\em rescheduling of a mid-term
  examination(s)} due to a religious, business-related,or scheduling
conflict must submit the Mid-term Examination Conflict form, together
with supporting documentation, to the School of Continuing Studies,
Client Services Office, at least two (2) weeks prior to the date of
the scheduled mid-term examination(s). Students who miss a mid-term
examination(s) due to medical reasons must complete and submit the
{\em Mid-term Examination Conflict form}, with supporting documentation,
within two (2) business days from the date of the missed
mid-term examination(s). The forms are available at
\url{https://www.mcgill.ca/continuingstudies/forms}.

Examination schedules are posted online approximately six (6) weeks
before the examination period begins. The exam schedule can be found
at
\url{https://www.mcgill.ca/continuingstudies/important-dates-exam-information}.

\newpage

\subsection{Resources }

\begin{description}
\item[Student Services]{Various services are available to Continuing
  Studies students. To learn more, visit \url{https://www.mcgill.ca/continuingstudies/getting-started}.}
\item[Students with Disabilities]{McGill is committed to providing an
  inclusive learning environment. If you experience barriers to
  learning in this course, do not hesitate to discuss them with me. If
  you have a special learning need or disability, you are encouraged
  to contact the Office for Students with Disabilities. To learn more,
  visit \url{https://www.mcgill.ca/osd}.}
 \item[Computer Labs]{Free access to computer labs is available at 688
   Sherbrooke (12th floor), MACES, the McLennan Library and other
   locations on campus.}
 \item[MACES]{The McGill Association of Continuing Education Students
   (MACES) is located at 3437 Peel, 2nd floor, tel.\ (514) 398-4974. To
   learn more, visit \url{https://maces.ca/}.}
\item[Career Advising and Transition Services]{Gain the clarity,
  skills, confidence and connections you need to succeed in your
  career thanks to the support provided by {\em Career Advising and
    Transition Services} (CATS). To learn more, visit
  \url{https://www.mcgill.ca/continuingstudies/career-advising-and-transition-services}.}
\item[myCourses]{Please check the myCourses site on a daily
  basis. Failure to do so may result in your missing important
  information. Neither absence from class nor failure to check
  myCourses is an acceptable excuse for being unaware of important
  course-related information.}
\item[Minerva]{For credit courses, access your personal student information online
  \url{https://horizon.mcgill.ca/pban1/twbkwbis.P_WWWLogin}. For
  issues related to student accounts, please call 514-398-7878.}
\item[Athena]{For non-credit courses, access your student profile and
  course information including online access to myCourses at
  \url{https://continuingstudies.mcgill.ca/portal/logon.do?method=load}.}
\item[IT Support]{Information related to online resources such as
  email,VPN, myCourses, etc. can be found at IT Services
  \url{https://www.mcgill.ca/it/}.}
\end{description}

\label{LastPage}



\end{document}
