\documentclass{article}

\usepackage[english]{babel}
\usepackage[utf8]{inputenc}
\usepackage{fancyhdr}
\usepackage{xcolor}
\usepackage{array}
\usepackage{helvet}
\usepackage{setspace}
\usepackage{graphicx}
\usepackage{longtable}

\PassOptionsToPackage{hyphens}{url}
\usepackage[hidelinks]{hyperref}
\usepackage{hyperref}
\usepackage{url}

\usepackage[left=18mm,right=18mm,top=20mm,bottom=40mm]{geometry}
\usepackage{titling}
%\setlength{\droptitle}{-25mm}

\setlength{\headheight}{25mm}
%\setlength{\footskip}{5mm}

\pagestyle{fancy}
\fancyhf{}
\lfoot{\textcolor{gray}{Course Outline, Technology \& Innovation}}
\lhead{{\includegraphics[width=120mm]{scs.png}}}
\rfoot{\textcolor{gray}{!!CODE!! (Section !!SECTION!!),  Page {\thepage} / \pageref{LastPage}}}

\setstretch{1.3}
\setlength{\parskip}{6pt}
\setlength{\parindent}{0pt}

\title{Course outline for !!CODE!! \\ {\sc !!NAME!!}}
\date{Section !!SECTION!! for !!TERM!! \\ !!CREDITS!! \\ !!KIND!!}
\author{{\bf McGill University School of Continuing Studies} \\
  Offering domain: {\em Technology \& Innovation} \\
  \\ Taught by \\ !!INSTRUCTOR!! }
\renewcommand{\familydefault}{\sfdefault} 

\begin{document}


\maketitle

\thispagestyle{fancy}

!!ERROR!!

\newpage

\setstretch{0.9}
\tableofcontents
\setstretch{1.2}

\newpage

\section{Course Information}

\begin{description}
\item[Office hours]{ !!HOURS!!}
!!ASSISTANT!!
\item[Description]{ !!DESCRIPTION!! }
\item[Course pre-requisite(s)]{!!PREREQ!!}
\item[Course co-requisite(s)]{!!COREQ!!}
\item[Contact hours]{!!CONTACT!!}
!!ASSIGNMENT!!
\item[Learning outcomes]{ !!OUTCOMES!! }
\item[Instructional methods]{ !!METHODS!! }

\end{description}


\vfill

\hrule
 
{\em The official version of this course outline is the version posted
  on myCourses on the day of the first class session.}

{\em In the event of extraordinary circumstances beyond the
  University's control, the content and/or evaluation scheme in this
  course is subject to change.}

\newpage

\paragraph{Note on email policy}

In accordance with McGill University’s official email
policy, the instructor {\em will not respond} to email messages sent
from {\bf outside} the McGill network (e.g., gmail or hotmail). As
well, while instructors appreciate that email correspondence is an
efficient method of communication, students should keep in mind that
instructors are not on call. Students should therefore not expect a
response to emails during the weekend or after 5:00 p.m., Mondays
through Fridays.

Prior to contacting instructors via email, students seeking
course-related information, including information about assignments,
should first attempt to locate that information by (a) consulting the
course syllabus or textbook, (b) consulting materials uploaded to the
course website (myCourses), or (c) consulting with other students in
the class. Generally speaking, during the week, a valid request
submitted via e-mail will receive a response within 36 hours,
exclusive of weekends and statutory holidays.

\newpage

\subsection{Evaluation}
\label{eval}

\begin{center}
  \renewcommand{\arraystretch}{1.2}
  \begin{longtable}{|>{\raggedright}p{36mm}|>{\raggedright}p{26mm}|r|p{76mm}|}
    \hline
        {\bf Name of assessment}
        & {\bf Due date}
        & {\bf \% of final grade} 
        & {\bf Assessment criteria} \\
        \hline
        \endhead        
        !!ITEMS!! \\
        \hline
  \end{longtable}
\end{center}

\newpage

\subsection{Weeks}

\begin{enumerate}
\item{!!CONTENT!!}
\end{enumerate}

\newpage

\section{Materials}

Please note that any electronic materials available on the LMS will be
accessible to students for six months after the conclusion of the
course.

\subsection{Hardware}

!!HARDWARE!!

!!ADMIN!!

\subsection{Software}

!!SOFTWARE!!

\subsection{Connectivity}

!!INTERNET!!

\newpage

!!READINGS!!

!!OPTIONAL!!

\section{{Additional Course Information \& Support}}

!!ADDITIONAL!!

\subsection{Student Support}

Please note that there is an {\bf MS Team for all Technology \& Innovation
students} you are encouraged to join: log into Microsoft Teams with
your McGill email and use the join code \texttt{e8i8f26} in the
lower-left corner (``Join or create a team'').

You can email
\href{mailto:help.ti.scs@mcgill.ca}{help.ti.scs@mcgill.ca} for
technical-academical assistance and
\href{mailto:ti.scs@mcgill.ca}{ti.scs@mcgill.ca} if you have
administrative academic questions.

\newpage

\section{Important Information}

\subsection{Land Acknowledgment}

McGill University is located on land, which has long served as a site
of meeting and exchange amongst Indigenous peoples, including the
Haudenosaunee and Anishinabeg nations. McGill honours, recognizes and
respects these nations as the traditional stewards of the lands and
waters on which we meet today.

\subsection{McGill Resources for Academic Success}

\begin{description}

\item[Inclusive Learning Environment]{McGill is committed to providing
  an inclusive and supportive learning environment. If you experience
  barriers to learning in this course, do not hesitate to discuss them
  with your instructor. If you have a special learning need or
  disability, you are encouraged to contact
  {\em Student Accessibility \& Achievement}.
  For more information, please visit
  \url{https://www.mcgill.ca/access-achieve/}.}
  
\item[Health and Wellness]{Student well-being is a priority for the
  University, the School (SCS), and the McGill Association of
  Continuing Education Students (MACES). Should you find yourself in
  need of support, please keep in mind that there are a number of
  resources available to help you. Many SCS students are automatically
  covered by the MACES Health and Dental Plan. For further details,
  please visit \url{https://maces.ca/}. In addition, eligible students
  will also be covered by a virtual healthcare service provided by
  {\em Dialogue} through MACES. The Dialogue service allows students
  to connect virtually with nurses and physicians in Canada via a
  mobile or web app. Both the MACES Health and Dental Plan and the
  Dialogue app include access to professional psychologists. MACES
  students also currently have free access to \texttt{Keep.meSAFE}, a
  psychological counselling service where students can speak to a
  counsellor in one of six languages. For more information, please
  visit
  \url{https://www.mcgill.ca/continuingstudies/student-services}.}
  
\item[McGill Writing Centre]{Writing well is key to both academic and
  professional success. The {\em McGill Writing Centre} (MWC) offers
  credit courses in academic and professional writing, and a tutorial
  service open to all McGill students. The tutorial service offers
  one-to-one sessions with seasoned instructors and experienced tutors
  who will work with you at any stage of the writing process.  For
  information about the availability of in-person and online
  appointments, please visit
  \url{https://www.mcgill.ca/mwc/tutorial-service}.}

\item[McGill Library]{Find a workshop, learn about library services,
  and reach out to your liaison librarian for research help at
  \url{https://www.mcgill.ca/library/orientation}.}
  
\item[Basic needs]{If you have difficulty affording food or if you
  lack a safe and stable place to live, and believe that these
  circumstances may affect your performance in this course, I
  encourage you to contact the Dean of Students
  \url{https://www.mcgill.ca/deanofstudents/}, who can connect you
  with support services. If you feel comfortable doing so, please
  let me know as well so we can discuss how I can best support your
  learning.}
  
\item[Workload management skills]{If you are feeling overwhelmed
  by your academic work and/or would like to further develop your
  time and workload management skills, don’t hesitate to seek
  support from Student Services
  \url{https://www.mcgill.ca/studentservices/}.}


\item[Learning Support Resources]{Consult resources from Teaching and Learning Services (TLS)
\url{https://www.mcgill.ca/tls/students/learning-resources} on topics
such as time management, study strategies, group work, exam prep, and
more. TLS also offers opportunities to connect with an academic peer
mentor through Stay on Track and to attend workshops. For further
individualized support check out the programs and resources from
Student Accessibility \& Achievement
\url{https://www.mcgill.ca/access-achieve/learner-support}.}

  
  
\end{description}


\subsection{Policy Against Sexual Harassment and Violence}

McGill University is committed to creating and sustaining a safe
environment through proactive, visible, accessible, and effective
approaches that seek to prevent and respond to sexual harassment and
sexual violence. McGill's Policy against Sexual Violence underlines
this commitment and ensures that procedures are in place to address
complaints. To learn more, visit \url{https://www.mcgill.ca/osvrse/}.

!!TRAINING!!

\subsection{Policy on Harassment and Discrimination}

McGill University is committed to promoting an equitable environment
where the fundamental dignity of all of its members is respected. The
objectives of {\em McGill's Policy on Harassment and Discrimination
Prohibited by Law} are to promote education and awareness about equity
issues and to ensure that procedures are in place to address
complaints. To learn more about McGill's policy, including how to
report a complaint, please visit
\url{https://www.mcgill.ca/how-to-report/}.


\subsection{Additional McGill policy statements}


\begin{description}
\item[Pronouns]{Please inform your instructional team if you would
  like them to refer to you by a different name than the name
  indicated in your student record or to inform them of your
  pronouns.}
  
\item[Recording privacy]{The instructor will notify you if part of
  a class is being recorded. By remaining in classes that are
  recorded, you agree to the recording, and you understand that
  your image, voice, and name may be disclosed to classmates. You
  also understand that recordings will be made available in
  myCourses to students registered in the course. Please consult
  the instructor if you have concerns about privacy and we can
  discuss possible measures that can be taken.}
  
\item[Sustainability]{McGill has policies on sustainability, paper
  use, and other initiatives to promote a culture of
  sustainability at McGill. See the Office of Sustainability
  \url{https://www.mcgill.ca/sustainability/}.}
  
\end{description}

\newpage

\subsection{Academic Conduct}

McGill University values academic integrity. Therefore, all students
must understand the meaning and consequences of cheating, plagiarism
and other academic offences under the {\em McGill Code of Student Conduct
  and Disciplinary Procedures}.

See \url{https://www.mcgill.ca/students/srr/honest/} for more
information.

\begin{description} 
\item[Academic Integrity]{Students are responsible for knowing
  McGill's rules and regulations concerning academic honesty, which
  can be found on the Student Rights and Responsibilities
  website. Violations of academic integrity undermine not only the
  value of honest students' work, but also the academic integrity of
  the University and the value of a McGill credential. The Student
  Rights and Responsibilities website provides resources that can help
  students avoid dishonest work, and an explanation of the
  disciplinary measures that go with it. 
  {\bf All newly admitted students must complete the Academic
    Integrity Tutorial (AIT) in Minerva during their first semester at
    McGill.} Failure to complete the tutorial will restrict the
  student from registering for courses in the following semester. The
  Tutorial can be accessed as follows: \texttt{Minerva / Student Menu
    / Academic Integrity Tutorial}.}
\item[Respectful and Professional Communication]{This course is
  designed to help you learn to communicate professionally both during
  your time at McGill and in your future workplaces. In keeping with
  McGill's policies on student rights and responsibilities, it is
  expected that during class discussions and small group interactions
  you will communicate constructively and respectfully. Sexist,
  racist, homophobic, ageist, and ableist expressions will not be
  tolerated in the classroom or during group meetings held outside of
  class.}
\item[Artificial Intelligence (AI) tools]{Note that any use of AI
  tools (such as ChatGPT) in assessments must comply with the
  instructions of each specific assessments. In general, AI tools
  should be clearly cited, along with any other consulted source.}
  
 \end{description}

To learn more about these policies, please consult 
\url{https://mcgill.ca/students/srr/policies-student-rights-and-responsibilities}.

Students {\bf may not record any class proceedings} or collect any
electronic data (including photos and videos) from class activities
without the express consent of the instructor. Instructor generated
course materials (e.g., handouts, notes, summaries, test questions,
etc.) are protected by law and may not be copied or distributed in any
form or in any medium without the explicit permission of the
instructor. Note that infringements of copyright can be subject to
followup by the University under the {\em Code of Student Conduct and
Disciplinary Procedures}.

\newpage

\subsection{Grades}

\subsubsection{Final Grades}

!!FINAL!!

A final grade appearing in other locations (for example, myCourses)
may be subject to change.  The School of Continuing Studies reserves
the right to correct mistakes.

!!GRADING!!

\paragraph{What does a final course grade of $J$ mean?}

A $J$ grade is a failing grade due to either (i) an unexcused absence
for an official final exam, or (ii) failure to submit required work
worth more than 20\% of the final grade for the course as a whole. A
$J$ grade is calculated as a failure.

\newpage

\subsection{Assessments}

Work submitted for evaluation as part of this course {\bf may be
  checked with text-matching software} within myCourses.

To learn more about assessment of student work, consult McGill's Policy on Assessment of Student Learning (PASL) at
\url{https://www.mcgill.ca/secretariat/files/secretariat/policy_on_assessment_of_student_learning.pdf}.

\subsubsection{Right to Submit in English or French}

In accord with McGill University's {\em Charter of Students' Rights},
students in this course have the right to submit in English or in
French any written work that is to be graded. This does not apply to
courses in which acquiring proficiency in a language is one of the
objectives.

Please reach out to the instructors and the assistants to inquire if
they are able to additionally advise you in other languages,
remembering that they are not required to do so. If you do share a
language, there is no impediment to using that in informal
discussions, as long as submitted work and any written agreements and
complaints are either in English or in French.

\subsubsection{Assignments}

Please refer to Section \ref{eval} concerning assignment submission
for this course. Unless this outline or another written document from
your instructor authorizes you to {\bf share} the assignment results
online or directly, please keep the questions and the responses {\em
  confidential}.

Legitimate exceptions for late submission of assignments include
documented medical, family, and personal emergencies, and observance
of holy days. Students planning to observe holy days (see
\url{https://www.mcgill.ca/importantdates/holy-days-0/policy-holy-days})
listed in the McGill calendar should notify the instructor by email no
later than two (2) weeks prior, and preferably at the beginning of the
course.

\subsubsection{Examinations}

If the course includes exams, students should not make other
commitments during a scheduled exam, which is indicated on your course
outline. Vacation plans do not constitute valid grounds for the
deferral or the rescheduling of examinations, tests or
assignments. See the School of Continuing Studies Calendar for the
regulations governing examinations, or go to
\url{https://www.mcgill.ca/continuingstudies/current-students/exams}.

Students who have a documented disability and require academic
accommodation must contact {\em Student Accessibility \&
  Achievement}. For information on Exam Accommodation, visit
\url{https://www.mcgill.ca/access-achieve/exams-accommodations/exams} and
\url{https://www.mcgill.ca/continuingstudies/scs-current-students/scs-student-services/scs-student-accessibility-achievement}.


Students may {\em request a deferral} of final examinations or timed
tests and assignments for medical reasons or out-of-town business
commitments, which must be justified in writing with a medical
certificate or company letter. To learn more, visit
\url{https://www.mcgill.ca/continuingstudies/exams-conflicts-deferrals-and-rereads}.

Students requesting the {\em rescheduling of a mid-term
  examination(s)} due to a religious, business-related,or scheduling
conflict must submit the Mid-term Examination Conflict form, together
with supporting documentation, to the School of Continuing Studies,
Client Services Office, at least two (2) weeks prior to the date of
the scheduled mid-term examination(s). Students who miss a mid-term
examination(s) due to medical reasons must complete and submit the
{\em Mid-term Examination Conflict form}, with supporting
documentation, within two (2) business days from the date of the
missed mid-term examination(s). The forms are available at
\url{https://www.mcgill.ca/continuingstudies/forms}.

Examination schedules are posted online approximately six (6) weeks
before the examination period begins. The exam schedule can be found
at
\url{https://www.mcgill.ca/continuingstudies/important-dates-exam-information}.

\subsubsection{Contesting Assessment Practices}
You can find the information about the contestation process on page 4 of the updated SCS course outline template. This template is available
on the "Teaching at SCS" webpage.

\subsection{Course evaluations}

!!COURSEEVAL!!

\newpage

\subsection{Resources }

\begin{description}
\item[Student Services]{Various services are available to Continuing
  Studies students. To learn more, visit \url{https://www.mcgill.ca/continuingstudies/getting-started}.}
\item[Students with Disabilities]{McGill is committed to providing an
  inclusive learning environment. If you experience barriers to
  learning in this course, do not hesitate to discuss them with me. If
  you have a special learning need or disability, you are encouraged
  to contact {\em Student Accessibility \& Achievement}.. To learn more,
  visit \url{https://www.mcgill.ca/access-achieve}.}
 \item[Computer Labs]{Free access to computer labs is available at 688
   Sherbrooke (12th floor), MACES, the McLennan Library and other
   locations on campus.}
 \item[MACES]{The McGill Association of Continuing Education Students
   (MACES) is located at 3437 Peel, 2nd floor, tel.\ (514) 398-4974. To
   learn more, visit \url{https://maces.ca/}.}
\item[Career Advising and Transition Services]{Gain the clarity,
  skills, confidence and connections you need to succeed in your
  career thanks to the support provided by {\em Career Advising and
    Transition Services} (CATS). To learn more, visit
  \url{https://www.mcgill.ca/continuingstudies/career-advising-and-transition-services}.}
\item[myCourses]{Please check the myCourses site on a daily
  basis. Failure to do so may result in your missing important
  information. Neither absence from class nor failure to check
  myCourses is an acceptable excuse for being unaware of important
  course-related information.}
!!PROFILE!!
\item[IT Support]{Information related to online resources such as
  email, VPN, myCourses, etc.\ can be found at IT Services
  \url{https://www.mcgill.ca/it/}.}
\end{description}

\label{LastPage}



\end{document}
